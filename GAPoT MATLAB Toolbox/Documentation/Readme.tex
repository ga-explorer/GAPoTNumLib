%% LyX 2.3.6 created this file.  For more info, see http://www.lyx.org/.
%% Do not edit unless you really know what you are doing.
\documentclass[english]{article}
\usepackage[T1]{fontenc}
\usepackage[latin9]{inputenc}
\usepackage{textcomp}
\usepackage{amstext}

\makeatletter

%%%%%%%%%%%%%%%%%%%%%%%%%%%%%% LyX specific LaTeX commands.
\newcommand{\lyxmathsym}[1]{\ifmmode\begingroup\def\b@ld{bold}
  \text{\ifx\math@version\b@ld\bfseries\fi#1}\endgroup\else#1\fi}


\makeatother

\usepackage{babel}
\begin{document}

\section{Installation and Usage}
\begin{enumerate}
\item Download the latest version of GAPoTNumLib from GitHub: https://github.com/ga-explorer/GAPoTNumLib
\item Open and Build the GAPoTNumLib solution using x64 Debug configuration.
Make sure the GAPoTNumLib.Framework project is the default project
of the solution
\item In the MATLAB toolbox open the file gapotInit.m and edit the variable
gapotAssemblyPath to be the path containing the GAPoTNumLib.Framework.exe
file
\item In MATLAB add the toolbox folder to the MATLAB path
\item You can find examples for using the toolbox in files gapotSample1.m,
gapotSample2.m, etc.
\end{enumerate}

\section{Representation of GAPoT Multivectors}

The design of GAPoTNumLib mainly targets the representation and manipulation
of sparse Euclidean multivectors containing elements of grades 0,1,
and 2. Contrast to most general purpose GA libraries, the dimension
of vectors in GAPoTNumLib is arbitrary, and can be in the range of
thouthands. Additionally, creation of GAPoTNumLib vectors is formulated
to be as close as possible to GAPoT symbolic representation of current
and voltage vectors. Other GAPoT multivectors are constructed using
the geometric product of GAPoT vectors.

A current or voltage vector in GAPoT is essentially a sum of $n$
polar phasors of the form:

\begin{eqnarray}
V & = & \sum_{i=1}^{n}\alpha_{i}\mathrm{exp}\left(\theta_{i}\sigma_{2i-1}\sigma_{2i}\right)\sigma_{2i-1}
\end{eqnarray}

Where $\alpha_{i},\theta_{i}$ are magnitudes and angles of individual
phasors. We can also re-write this into two equivalent forms. The
rectangular phasor form is:

\begin{eqnarray}
V & = & \sum_{i=1}^{n}\alpha_{i}\left(\mathrm{cos}\theta_{i}+\mathrm{sin}\theta_{i}\sigma_{2i-1}\sigma_{2i}\right)\sigma_{2i-1}\nonumber \\
 & = & \sum_{i=1}^{n}\left(x_{i}+y_{i}\sigma_{2i-1}\sigma_{2i}\right)\sigma_{2i-1}
\end{eqnarray}

Where $x_{i}=\alpha_{i}\mathrm{cos}\theta_{i},y_{i}=\alpha_{i}\mathrm{sin}\theta_{i}$
are cartesian components of the phasor. The third form is the most
similar to traditional representation of multivectors in GA libraries
as a sum of scaled basis blades:

\begin{eqnarray}
V & = & \sum_{i=1}^{n}\left(x_{i}\sigma_{2i-1}+y_{i}\sigma_{2i-1}\sigma_{2i}\sigma_{2i-1}\right)\nonumber \\
 & = & \sum_{i=1}^{n}\left(x_{i}\sigma_{2i-1}-y_{i}\sigma_{2i}\right)\nonumber \\
 & = & \sum_{i=1}^{2n}v_{i}\sigma_{i}
\end{eqnarray}

Where $v_{i}=x_{i}=\alpha_{i}\mathrm{cos}\theta_{i}$ for $i=1,3,5,\ldots,2n-1$,
and $v_{i}=-y_{i}=-\alpha_{i}\mathrm{sin}\theta_{i}$ for $i=2,4,6,\ldots,2n$.

In GAPoTNumLib, a simple textual representation can be used to construct
GAPoT vectors using either 3 forms: sum of polar phasors, rectangular
phasors, or terms. Internally, however, all GAPoT vectors are stored
as a sparse list of terms and other forms are composed and displayed
to the user on demand. The following are examples for the textual
representation of vectors in GAPoTNumLib:
\begin{itemize}
\item \texttt{'-1.3<1>, 1.2<3>, -4.6<6>'} represents a GAPoT vector in sum
of terms form: $-1.3\sigma_{1}+1.2\sigma_{3}-4.6\sigma_{6}$
\item \texttt{'p(233.92, \textminus 90) <1,2>, p(-120, 30) <3,4>'} represents
a GAPoT vector in sum of polar phasors form: $233.92e^{-90^{\lyxmathsym{\textdegree}}\ensuremath{\sigma_{1}\sigma_{2}}}\ensuremath{\sigma_{1}}-120\ensuremath{e^{30^{\lyxmathsym{\textdegree}}\ensuremath{\sigma_{3}\sigma_{4}}}}\sigma_{3}$
\item \texttt{'r(10, 20) <1,2>, r(30, 0) <3,4>'} represents a GAPoT vector
in sum of rectangular phasors form: $\left(10+20\sigma_{1}\sigma_{2}\right)\sigma_{1}+\left(30+0\sigma_{3}\sigma_{4}\right)\sigma_{3}=\left(10\sigma_{1}-20\sigma_{2}\right)+\left(30\sigma_{3}\right)$
which is equivalent to \texttt{'10<1>, -20<2>, 30<3>'}
\item \texttt{'1<1>, r(10, 20) <2,3>, p(-120, 30) <4,5>'} represents a GAPoT
vector in sum of phasors form: $1\sigma_{1}+\left(10+20\sigma_{3}\sigma_{4}\right)\sigma_{3}-120e^{30^{\lyxmathsym{\textdegree}}\ensuremath{\sigma_{5}\sigma_{6}}}\ensuremath{\sigma_{5}}$
\end{itemize}
The geometric product of two GAPoT vectors is a multivector containing
only grade 0 and 2 elements. In GAPoTNumLib such multivectors are
called biversors, as they are the geometric product of two vectors.
The user can also create a biversor from a textual representation
such as \texttt{'3<>, -2<1,2>, 4<3,4>'} which represents the multivector
$3-2\sigma_{1}\sigma_{2}+4\sigma_{3}\sigma_{4}$.

\section{GAPoTNumLib Classes}

In the .NET solution, the user can find several classes to represent
GAPoT multivectors under the \texttt{GAPoTNumLib.GAPoT} namespace
as follows:
\begin{itemize}
\item \texttt{GaPoTNumVector} is the class used to represent GAPoT vectors,
typically holding currents and voltages. The main operations this
class provides include setting, getting, and adding terms, polar phasors,
and rectangular phasors. The user can also add and subtract two GAPoT
vectors, compute their geometric product, find the negative, norm,
squared norm, reverse, and inverse of a GAPoT vector. In addition,
several methods for displaying the GAPoT vector in various text and
LaTeX formats exist.
\item \texttt{GaPoTNumBiversor} is used to represent GAPoT biversors; sparse
multivectors which only contains elements of grades 0 and 2 typically
representing power and impedance. The class provides methods for setting,
getting, and adding individual terms of grade 0 and 2. The user can
also compute the negative, norm, squared norm, reverse, and inverse
of biversors. There are several methods for extracting power quantities
from the biversor such as asctive, non-active, reactive, fundamental
reactive, and harmful power parts. In addition, several methods for
displaying the GAPoT biversor in various text and LaTeX formats exist.
\end{itemize}
For the elements of a GAPoT vector, there are three classes that can
be used to hold information on terms and phasors:
\begin{itemize}
\item \texttt{GaPoTNumVectorTerm} holds a single term vector.
\item \texttt{GaPoTNumPolarPhasor} holds a single polar phasor vector.
\item \texttt{GaPoTNumRectPhasor} holds a single rectangular phasor vector.
\end{itemize}

\section{Operations on Multivectors}

\subsection{Construction and Update Operations}

\subsection{Mathematical Operations}

\subsection{Text Operations}

\subsection{MATLAB Interoperability}
\end{document}
